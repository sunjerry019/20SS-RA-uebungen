\newcommand{\boxy}[2][yellow]{\mathchoice%
  {\pgfsetfillopacity{0.3}\colorbox{#1}{\pgfsetfillopacity{1}$\displaystyle#2$}}%
  {\pgfsetfillopacity{0.3}\colorbox{#1}{\pgfsetfillopacity{1}$\textstyle#2$}}%
  {\pgfsetfillopacity{0.3}\colorbox{#1}{\pgfsetfillopacity{1}$\scriptstyle#2$}}%
  {\pgfsetfillopacity{0.3}\colorbox{#1}{\pgfsetfillopacity{1}$\scriptscriptstyle#2$}}}%

\let\bnot\xoverline
\let\bnor\downarrow
\newcommand{\xorexpanded}[2]{\bnot{#1}#2 \lor #1\bnot{#2}}
\newcommand{\xor}[0]{\nleftrightarrow}
\newcommand{\linenum}[1]{\textcolor{RedViolet}{\texttt{#1}}}
\tikzstyle{every picture}+=[remember picture]

\begin{enumerate}[label={[OH\arabic*]},start=11]
    \item
        \begin{enumerate}
            \makeatletter
                \setlength{\leftmargins}{\@totalleftmargin}
            \makeatother

            \item 
                \begin{enumerate}
                    \item Zeile \linenum{17}: Der aktuelle Buchstabe des Strings \texttt{needle} wird in ein Register geladen.
                    \item Zeile \linenum{72}: Die Anzahl der gezählten Vokale wird auf der Konsole ausgegeben.
                    \item Zeile \linenum{37}: Erhöhe die Anzahl der gezählten Vokale um eins.
                    \item Zeile \linenum{18}: Führe einen Sprung durch, sofern alle Buchstaben des Strings needle betrachtet wurden.
                \end{enumerate}

            % https://quantixed.org/2018/10/23/new-lexicon-how-to-add-a-custom-minted-lexer-in-overleaf/
            % breaklines
            \item \blanko
                { \small \inputminted[linenos,firstnumber=last,autogobble,xleftmargin=-\leftmargins,frame=leftline,framesep=10pt]{mipslexer.py:MIPSLexer -x}{vokale.s} }
        \end{enumerate}
\end{enumerate}