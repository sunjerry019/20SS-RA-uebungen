\newcommand{\boxy}[2][yellow]{\mathchoice%
  {\pgfsetfillopacity{0.3}\colorbox{#1}{\pgfsetfillopacity{1}$\displaystyle#2$}}%
  {\pgfsetfillopacity{0.3}\colorbox{#1}{\pgfsetfillopacity{1}$\textstyle#2$}}%
  {\pgfsetfillopacity{0.3}\colorbox{#1}{\pgfsetfillopacity{1}$\scriptstyle#2$}}%
  {\pgfsetfillopacity{0.3}\colorbox{#1}{\pgfsetfillopacity{1}$\scriptscriptstyle#2$}}}%

\let\bnot\xoverline
\let\bnor\downarrow
\newcommand{\xorexpanded}[2]{\bnot{#1}#2 \lor #1\bnot{#2}}
\newcommand{\xor}[0]{\nleftrightarrow}
% \tikzstyle{every picture}+=[remember picture]
\newcommand*{\VA}[0]{\textcolor{blue}{VA}}

% https://tex.stackexchange.com/a/401771
\pgfdeclarearrow{
    name=Midcircle,
    parameters= {\the\pgfarrowlength},  
    setup code={
        \pgfarrowssettipend{0.25\pgfarrowlength}
        \pgfarrowssetlineend{-0.25\pgfarrowlength}
        \pgfarrowlinewidth=\pgflinewidth
        \pgfarrowssavethe\pgfarrowlength
    },
    drawing code={
        \pgfpathcircle{\pgfpoint{0.25\pgfarrowlength}{0pt}}{0.5\pgfarrowlength}
        \pgfusepathqfill
    },
    defaults = { length = 2pt }
}
\pgfdeclarearrow{
    name=MidcircleBig,
    parameters= {\the\pgfarrowlength},  
    setup code={
        \pgfarrowssettipend{0.25\pgfarrowlength}
        \pgfarrowssetlineend{-0.25\pgfarrowlength}
        \pgfarrowlinewidth=\pgflinewidth
        \pgfarrowssavethe\pgfarrowlength
    },
    drawing code={
        \pgfpathcircle{\pgfpoint{0.25\pgfarrowlength}{0pt}}{0.5\pgfarrowlength}
        \pgfusepathqfill
    },
    defaults = { length = 4pt }
}

% https://tex.stackexchange.com/a/228533
\makeatletter
\newcommand\captionof[1]{\def\@captype{#1}\caption}
\makeatother

\begin{enumerate}[label={[OH\arabic*]},start=12]
    \item
        \begin{enumerate}[label={(\alph*)}]
            \item \blanko
                \vspace{-\baselineskip}
                % https://tex.stackexchange.com/a/11706/116525
                \begin{center}
                    \begin{tabular}{c *{5}{@{\,}c}}
                              & $x_4$ & $x_3$ & $x_2$ & $x_1$ & $x_0$ \\
                        $+$   & $y_4$ & $y_3$ & $y_2$ & $y_1$ & $y_0$ \\
                        \hline
                        $U_{\text{out}}$ & $R_4$ & $R_3$ & $R_2$ & $R_1$ & $R_0$
                    \end{tabular}
                \end{center}
                \begin{align*}
                    U_{\text{out}} = &~ x_4y_4 \\
                    &+ (x_4 + y_4)\,x_3y_3 \\
                    &+ (x_4 + y_4)(x_3+y_3)\,x_2y_2 \\
                    &+ (x_4 + y_4)(x_3+y_3)(x_2+y_2)\,x_1y_1 \\
                    &+ (x_4 + y_4)(x_3+y_3)(x_2+y_2)(x_1+y_1)\,x_0y_0 \\
                    &+ (x_4 + y_4)(x_3+y_3)(x_2+y_2)(x_1+y_1)(x_0 + y_0)\,U_\text{in} \\
                \end{align*}
            \newpage
            \item Es wird angenommen, dass man den eingehenden Übertrag auch betrachten muss:
                \makeatletter
                    \setlength{\leftmargins}{\@totalleftmargin}
                \makeatother

                \vspace{\baselineskip} 
                \hspace{-\leftmargins}\hspace{-1cm}{\scriptsize
\begin{tikzpicture}[circuit logic US,scale=0.7, every node/.style={scale=0.7},
	fulladder/.style={draw, minimum width=2cm,minimum height=1.5cm,
    label={[anchor=west]left:\textcolor{black!65}{$U_\text{out}$}},
    label={[anchor=south]below:\textcolor{black!65}{$R$}},
    label={[anchor=east]right:\textcolor{black!65}{$U_\text{in}$}},
    label={[anchor=north]65:\textcolor{black!65}{$y$}},
    label={[anchor=north]115:\textcolor{black!65}{$x\vphantom{y}$}},
    }] %scale=0.6, every node/.style={scale=0.6}

    % https://tex.stackexchange.com/a/277983

	% and and ors for inputs directly
	\node[and gate,inputs={nn}, point left] (and4) at (0,-1)    {};
	\node[or gate,inputs={nn}, point left]  (or4)  at (0,-2)    {};
	\node[and gate,inputs={nn}, point left] (and3) at (0,-3)    {};
	\node[or gate,inputs={nn}, point left]  (or3)  at (0,-4)    {};
	\node[and gate,inputs={nn}, point left] (and2) at (0,-5)    {};
	\node[or gate,inputs={nn}, point left]  (or2)  at (0,-6)    {};
	\node[and gate,inputs={nn}, point left] (and1) at (0,-7)    {};
	\node[or gate,inputs={nn}, point left]  (or1)  at (0,-8)    {};
	\node[and gate,inputs={nn}, point left] (and0) at (0,-9)    {};
	\node[or gate,inputs={nn}, point left]  (or0)  at (0,-10)   {};

	% summation and gates
	\node[and gate,inputs={nnnnnn}, point down] (and-comb0) at (-1.5,-11.5)    {};
	\node[and gate,inputs={nnnnn}, point down]  (and-comb1) at (-3.5,-10)      {};
	\node[and gate,inputs={nnnn}, point down]   (and-comb2) at (-5.5,-8.5)     {};
	\node[and gate,inputs={nnn}, point down]    (and-comb3) at (-7.5,-7.0)     {};
	\node[and gate,inputs={nn}, point down]     (and-comb4) at (-9.5,-5.5)     {};

	% Final or
	\node[or gate,inputs={nnnnnn}, point down]  (or-final)  at (-6.5,-13)   {};

	% VA
	\node[fulladder] (z4) at (2,-11.5)   {\large \VA};
	\node[fulladder] (z3) at (4.5,-11.5) {\large \VA};
	\node[fulladder] (z2) at (7,-11.5)   {\large \VA};
	\node[fulladder] (z1) at (9.5,-11.5) {\large \VA};
	\node[fulladder] (z0) at (12,-11.5)  {\large \VA};

	% labels
	%                    angle:dist
	\draw[] (z4.115) --++(90:10.5) node [above] (x4) {\large $x_4\vphantom{y_4}$};
	\draw[] (z4.65)  --++(90:10.5) node [above] (y4) {\large $y_4$};
	\draw[] (z3.115) --++(90:10.5) node [above] (x3) {\large $x_3\vphantom{y_3}$};
	\draw[] (z3.65)  --++(90:10.5) node [above] (y3) {\large $y_3$};
	\draw[] (z2.115) --++(90:10.5) node [above] (x2) {\large $x_2\vphantom{y_2}$};
	\draw[] (z2.65)  --++(90:10.5) node [above] (y2) {\large $y_2$};
	\draw[] (z1.115) --++(90:10.5) node [above] (x1) {\large $x_1\vphantom{y_1}$};
	\draw[] (z1.65)  --++(90:10.5) node [above] (y1) {\large $y_1$};
	\draw[] (z0.115) --++(90:10.5) node [above] (x0) {\large $x_0\vphantom{y_0}$};
	\draw[] (z0.65)  --++(90:10.5) node [above] (y0) {\large $y_0$};
	\node[right of=y0] (u0) {\large $U_\text{in}$};

	% https://tex.stackexchange.com/a/277983

	% LINES
	%                  first vert then hori
	% https://stuff.mit.edu/afs/athena/contrib/tex-contrib/beamer/pgf-1.01/doc/generic/pgf/version-for-tex4ht/en/pgfmanualse9.html

	% Carries
	\draw[] (u0.south) |- (z0.east);
	\draw[] (z0.west) -- (z1.east);
	\draw[] (z1.west) -- (z2.east);
	\draw[] (z2.west) -- (z3.east);
	\draw[] (z3.west) -- (z4.east);
	\draw[] (z4.west) --++(180:0.5);

	% To first stage
	\draw[Midcircle-] (x4 |- and4.input 2) -- (and4.input 2);
	\draw[Midcircle-] (y4 |- and4.input 1) -- (and4.input 1);
	\draw[Midcircle-] (x4 |- or4.input 2) -- (or4.input 2);
	\draw[Midcircle-] (y4 |- or4.input 1) -- (or4.input 1);

	\draw[Midcircle-] (x3 |- and3.input 2) -- (and3.input 2);
	\draw[Midcircle-] (y3 |- and3.input 1) -- (and3.input 1);
	\draw[Midcircle-] (x3 |- or3.input 2) -- (or3.input 2);
	\draw[Midcircle-] (y3 |- or3.input 1) -- (or3.input 1);

	\draw[Midcircle-] (x2 |- and2.input 2) -- (and2.input 2);
	\draw[Midcircle-] (y2 |- and2.input 1) -- (and2.input 1);
	\draw[Midcircle-] (x2 |- or2.input 2) -- (or2.input 2);
	\draw[Midcircle-] (y2 |- or2.input 1) -- (or2.input 1);

	\draw[Midcircle-] (x1 |- and1.input 2) -- (and1.input 2);
	\draw[Midcircle-] (y1 |- and1.input 1) -- (and1.input 1);
	\draw[Midcircle-] (x1 |- or1.input 2) -- (or1.input 2);
	\draw[Midcircle-] (y1 |- or1.input 1) -- (or1.input 1);

	\draw[Midcircle-] (x0 |- and0.input 2) -- (and0.input 2);
	\draw[Midcircle-] (y0 |- and0.input 1) -- (and0.input 1);
	\draw[Midcircle-] (x0 |- or0.input 2) -- (or0.input 2);
	\draw[Midcircle-] (y0 |- or0.input 1) -- (or0.input 1);

	% To and stage
	%4
		\draw[] (or4.output)  -| (and-comb4.input 2);
		\draw[] (and3.output) -| (and-comb4.input 1);
	%3
		\draw[Midcircle-] (or4.output -| and-comb3.input 3) -- (and-comb3.input 3);
		\draw[] (or3.output)  -| (and-comb3.input 2);
		\draw[] (and2.output) -| (and-comb3.input 1);
	%2
		\draw[Midcircle-] (or4.output -| and-comb2.input 4) -- (and-comb2.input 4);
		\draw[Midcircle-] (or3.output -| and-comb2.input 3) -- (and-comb2.input 3);
		\draw[] (or2.output)  -| (and-comb2.input 2);
		\draw[] (and1.output) -| (and-comb2.input 1);
	%1
		\draw[Midcircle-] (or4.output -| and-comb1.input 5) -- (and-comb1.input 5);
		\draw[Midcircle-] (or3.output -| and-comb1.input 4) -- (and-comb1.input 4);
		\draw[Midcircle-] (or2.output -| and-comb1.input 3) -- (and-comb1.input 3);
		\draw[] (or1.output)  -| (and-comb1.input 2);
		\draw[] (and0.output) -| (and-comb1.input 1);
	%0
		\draw[Midcircle-] (or4.output -| and-comb0.input 6) -- (and-comb0.input 6);
		\draw[Midcircle-] (or3.output -| and-comb0.input 5) -- (and-comb0.input 5);
		\draw[Midcircle-] (or2.output -| and-comb0.input 4) -- (and-comb0.input 4);
		\draw[Midcircle-] (or1.output -| and-comb0.input 3) -- (and-comb0.input 3);
		\draw[] (or0.output)  -| (and-comb0.input 2);
		\draw[Midcircle-] ($(u0)-(0,10.4)$) -| (and-comb0.input 1);

	% To final or stage
	\draw[] (and4.output) -| (-10.5,-12) -| (or-final.input 6);
	\draw[] (and-comb4.output) --++(270:6) -| (or-final.input 5);
	\draw[] (and-comb3.output) --++(270:4.3) -| (or-final.input 4);
	\draw[] (and-comb2.output) --++(270:2.735) -| (or-final.input 3);
	\draw[] (and-comb1.output) --++(270:1.5) -| (or-final.input 2);
	\draw[] (and-comb0.output) --++(270:0.25) -| (or-final.input 1);

	% Output
	\draw[] (z4.south) --++(270:2) node [below] (r4) {\large $R_4$};
	\draw[] (z3.south) --++(270:2) node [below] (r3) {\large $R_3$};
	\draw[] (z2.south) --++(270:2) node [below] (r2) {\large $R_2$};
	\draw[] (z1.south) --++(270:2) node [below] (r1) {\large $R_1$};
	\draw[] (z0.south) --++(270:2) node [below] (r0) {\large $R_0$};
	\draw[] (or-final.output) --++(270:0.7) node [below] (u1) {\large $U_\text{out}$};
\end{tikzpicture}} 
                \vspace{\baselineskip} 
            \item \label{qn:ha}
                Setze $12141043$ in der Formel:
                \begin{align*}
                    z &= \left(\left(\frac{\text{Matr. Nr}}{2}\cdot 10\right) \mod 1000\right) + 25 \\
                    &= \left(\left(\frac{12141043}{2}\cdot 10\right) \mod 1000\right) + 25 \\
                    &=\left(60705215 \mod 1000\right) + 25 \\
                    &= 215 + 25\\
                    &= 240
                \end{align*}
                % Da die Aufgabestellung fordert keine explizite Verwendung von Halbaddierer an, gehen wir davon aus, dass die Addition zweier $240$-Bit Dualzahlen ausschließlich aus dem oben gezeichneten Carry-Look-Ahead-Addierer realisiert werden ist (d.h. Es steht keinen Halbaddierer zur Verfügung).

                % Wir gehen davon aus, dass wir nur die oben gezeichnete Carry-Look-Ahead-Addierer  verwenden (d.h. Es steht keinen Halbaddierer zur Verfügung).
                Ein Carry-Look-Ahead-Addierer mit einer Größe der Bit-Gruppen von $g=5$ braucht insgesamt
                \begin{equation}
                    3\times\SI{10}{\pico\second} = \SI{30}{\pico\second}
                \end{equation}
                für $U_\text{out}$ ($3$ Stufen) und 
                \begin{equation}
                    5\times\SI{70}{\pico\second} = \SI{350}{\pico\second}
                \end{equation}
                für $R_0$ bis $R_4$ (fünf Volladdierer, jeweils mit $T_\text{VA} = \SI{70}{\pico\second}$). Da es mehrere Carry-Look-Ahead-Addierer hintereinander geschaltet werden, ist es egal, ob der erste Carry-Look-Ahead-Addierer einen Volladdier oder einen Halbaddier für die erste Ziffer $R_0$ benutzt. Wir brauchen also insgesamt:
                \begin{align*}
                    T_{\text{CLA}} &= \left(\frac{240}{5} - 1\right) T_{(U\text{out})} + \SI{350}{\pico\second} \\
                    &= 47 \times \SI{30}{\pico\second} + \SI{350}{\pico\second} \\
                    &= \boxy{\SI{1760}{\pico\second}}
                \end{align*}
                Für ein Ripple-Carry-Addiernetz brauchen wir aber einen Halbaddierer für die erste Ziffer. Da es in der Aufgabestellung aber nicht gegeben war, berechnen wir zunächst die Verzögerung eines Halbaddierers der Form:

                \begin{minipage}{\linewidth}
                    \vspace{\baselineskip}
                    \centering
                    \begin{tikzpicture}[circuit logic US, scale=0.9, every node/.style={scale=0.9}]
                        \node[and gate,inputs={ni}, point down] (and1) at (0,-1)    {};
                        \node[and gate,inputs={in}, point down] (and2) at (1,-1)    {};
                        \node[and gate,inputs={nn}, point down] (and3) at (2,-1)    {};

                        \node[or gate,inputs={nn}, point down] (or) at ($(and1)!0.5!(and2)+(0,-1)$)    {};

                        \draw[] (and3.input 1) |- (-1, 0.2) node[left] (x) {$x$};
                        \draw[] (and3.input 2) |- (-1,-0.2) node[left] (y) {$y$};
                        \draw[-Midcircle] (and1.input 2) |- ( y -| and1.input 2);
                        \draw[-Midcircle] (and1.input 1) |- ( x -| and1.input 1);
                        \draw[-Midcircle] (and2.input 2) |- ( y -| and2.input 2);
                        \draw[-Midcircle] (and2.input 1) |- ( x -| and2.input 1);

                        \draw[] (and1.output) |- ($(or.input 2)+(0,0.2)$) -- (or.input 2);
                        \draw[] (and2.output) |- ($(or.input 1)+(0,0.2)$) -- (or.input 1);

                        \draw[] (or.output)   --++(270:0.3) node [below] (r) {$R$};
                        \node[] (u) at (r -| and3.output) {$U$};
                        \draw[] (and3.output) -- (u) ;
                    \end{tikzpicture}
                    \captionof{figure}{Halbaddierer}
                    \label{fig:ha}
                    \vspace{\baselineskip}
                \end{minipage}

                Angenommen, dass ein \texttt{NOT}-Gatter auch eine Verzögerung von $\SI{10}{\pico\second}$ verursacht, dann haben wir:
                \begin{equation}
                    T_{\text{HA}} = 3 \times \SI{10}{\pico\second} = \SI{30}{\pico\second} 
                \end{equation}
                Somit braucht ein Ripple-Carry-Addiernetz, das zwei $240$-stellige Dualzahlen addieren kann, insgesamt:
                \begin{align*}
                    T_{\text{RC}} &= (240 - 1)\times T_{\text{VA}} + T_{\text{HA}} \\
                    &=239 \times \SI{70}{\pico\second} + \SI{30}{\pico\second} \\
                    &= \boxy{\SI{16760}{\pico\second}}
                \end{align*}
                Stehen es nur Volladdierer zur Verfügung, dann braucht ein Ripple-Carry-Addiernetz insgesamt:
                \begin{align*}
                    T'_{\text{RC}} &= 240\times T_{\text{VA}} \\
                    &=240 \times \SI{70}{\pico\second}\\
                    &= \boxy{\SI{16800}{\pico\second}}
                \end{align*}
        \end{enumerate}
    \item 
        \begin{enumerate}
            \item Die Funktionstafel für die gewünschte Schaltung ist gegeben durch:
                \begin{equation*}
                    \begin{tabu}{ccc|cc}
                        \toprule
                        p & x_1 & x_0 & y_1 & y_0 \\
                        \midrule
                        0 & 0 & 0 & D & D \\
                        0 & 0 & 1 & D & D \\
                        0 & 1 & 0 & D & D \\
                        0 & 1 & 1 & D & D \\
                        1 & 0 & 0 & 1 & 1 \\
                        1 & 0 & 1 & 0 & 0 \\
                        1 & 1 & 0 & 0 & 1 \\
                        1 & 1 & 1 & 1 & 0 \\
                        \bottomrule
                    \end{tabu}
                \end{equation*}
            \item Da die Ausgänge der Schaltung für $p = 0$ egal sind, betrachten wir nur die untere Hälfte der Funktionstabelle und konstruieren die Schaltung so, dass die Schaltung immer das Predecessor liefert, egal $p = 0$ oder $p=1$ ist. Als boolsche Funktionen sind $y_0$ und $y_1$ somit gegeben durch:
            \begin{align}
                y_0 &= \bnot{x_1}\bnot{x_0} + x_1\bnot{x_0} = \bnot{x_0} \\
                y_1 &= \bnot{x_1}\bnot{x_0} + x_1x_0
            \end{align}
            In einem Halbaddierer mit Eingänge $a$ und $b$ sind die Ausgänge $R$ und $U$ gegeben durch:
            \begin{align}
                R &= \bnot{a}b + a\bnot{b} \\
                U &= ab 
            \end{align}
            Ist nun $a = \bnot{x_0}$ und $b = x_1$, dann gilt:
            \begin{align}
                R =  \bnot{\bnot{x_0}}x_1 + \bnot{x_0}\bnot{x_1}= \bnot{x_1}\bnot{x_0} + x_1x_0 = y_1
            \end{align}
            Wir benutzten nun den in Abbildung \ref{fig:ha} aus Teilaufgabe \ref{qn:ha} beschriebenen Halbaddierer mit der Schaltsymbol:
            \begin{center}
                \begin{tikzpicture}[circuit logic US, scale=0.9, every node/.style={scale=0.9},halfadder/.style={draw, minimum width=2cm,minimum height=1.5cm,
                    label={[anchor=east]340:\textcolor{black!50}{$U$}},
                    label={[anchor=east]20:\textcolor{black!50}{$R$}},
                    label={[anchor=west]160:\textcolor{black!50}{$a$}},
                    label={[anchor=west]200:\textcolor{black!50}{$b$}},
                    }]

                    \node[halfadder] (ha) at (0,0)  {\large HA};
                \end{tikzpicture}
            \end{center}
            Die Predecessor-Funktion kann somit durch Halbaddierer und \texttt{NOT}-Gatter realisert werden:
            \begin{center}
                \begin{tikzpicture}[circuit logic US, scale=0.9, every node/.style={scale=0.9},halfadder/.style={draw, minimum width=2cm,minimum height=1.5cm,
                    label={[anchor=east]340:\textcolor{black!50}{$U$}},
                    label={[anchor=east]20:\textcolor{black!50}{$R$}},
                    label={[anchor=west]160:\textcolor{black!50}{$a$}},
                    label={[anchor=west]200:\textcolor{black!50}{$b$}},
                    }] 

                    \node[halfadder] (ha1) at (0,0)  {\large HA};
                    \node[not gate,inputs={n}, point right] (not) at ($(ha1.west)+(-2,1)$) {};

                    \draw[-MidcircleBig] (ha1.160)  -| ($(not.output)+(0.3,0)$);

                    % labels
                    \node[] (x0)   at ($(not.input)+(-2,0)$) {$x_0$};
                    \node[] (pp)   at ($(x0)+(0,0.7)$) {$p$};
                    \node[] (p)    at (pp) {$\hphantom{x_0}$};
                    \node[] (x1)   at (x0 |- ha1.200) {$x_1$};

                    \draw[] (p)  --++(0:1);
                    \draw[] (x0) -- (not.input);
                    \draw[] (x1) -- (ha1.200);

                    \draw[] (ha1.340) --++(0:0.5);
                    \draw[] (ha1.20)  --++(0:1.5) node[right] (y1) {$y_1$};
                    \node[] (y0) at (not -| y1) {$y_0$};
                    \draw[] (not.output) -- (y0);

                    % Box
                    \node[] (bottomright) at (x1 -| y1) {};
                    \draw[dashed] ($(p)+(0.7,0.5)$) rectangle ($(bottomright)+(-0.7,-0.5)$);
                \end{tikzpicture}
            \end{center}
            Wenn man aber unbedingt die $p$-Eingang benutzen wollen, dann kann die Predecessor-Funktion wie folgt realisert werden:
            \begin{center}
                \begin{tikzpicture}[circuit logic US, scale=0.9, every node/.style={scale=0.9},halfadder/.style={draw, minimum width=2cm,minimum height=1.5cm,
                    label={[anchor=east]340:\textcolor{black!50}{$U$}},
                    label={[anchor=east]20:\textcolor{black!50}{$R$}},
                    label={[anchor=west]160:\textcolor{black!50}{$a$}},
                    label={[anchor=west]200:\textcolor{black!50}{$b$}},
                    }] 

                    \node[halfadder] (ha1) at (0,0)  {\large HA1};
                    \node[halfadder] (ha2) at (2.5,1.5)  {\large HA2};
                    \node[not gate,inputs={n}, point right] (not) at ($(ha2.200)+(-3.8,0)$) {};

                    \draw[]           (not.output) -- (ha2.200);
                    \draw[-MidcircleBig] (ha1.160)    -| ($(not.output)+(0.3,0)$);

                    % labels
                    \node[] (x0)   at ($(not.input)+(-2,0)$) {$x_0$};
                    \node[] (pp)   at (x0 |- ha2.160) {$p$};
                    \node[] (p)    at (x0 |- ha2.160) {$\hphantom{x_0}$};
                    \node[] (x1)  at (x0 |- ha1.200) {$x_1$};

                    \draw[] (p)  -- (ha2.160);
                    \draw[] (x0) -- (not.input);
                    \draw[] (x1) -- (ha1.200);

                    \draw[] (ha1.340) --++(0:0.5);
                    \draw[] (ha2.20)  --++(0:0.5);
                    \draw[] (ha1.20)  --++(0:4) node[right] (y1) {$y_1$};
                    \node[] (y0) at (ha2.340 -| y1) {$y_0$};
                    \draw[] (ha2.340) -- (y0);

                    % Box
                    \node[] (bottomright) at (x1 -| y1) {};
                    \draw[dashed] ($(p)+(0.7,0.5)$) rectangle ($(bottomright)+(-0.7,-0.5)$);
                \end{tikzpicture}
            \end{center}
            mit $y_0 = (p\land\bnot{x_0})$, was $\bnot{x_0}$ liefert für $p = 1$.

        \end{enumerate}
\end{enumerate}