\newcommand{\boxy}[2][yellow]{\mathchoice%
  {\pgfsetfillopacity{0.3}\colorbox{#1}{\pgfsetfillopacity{1}$\displaystyle#2$}}%
  {\pgfsetfillopacity{0.3}\colorbox{#1}{\pgfsetfillopacity{1}$\textstyle#2$}}%
  {\pgfsetfillopacity{0.3}\colorbox{#1}{\pgfsetfillopacity{1}$\scriptstyle#2$}}%
  {\pgfsetfillopacity{0.3}\colorbox{#1}{\pgfsetfillopacity{1}$\scriptscriptstyle#2$}}}%

\let\not\xoverline
\let\nor\downarrow
\newcommand{\xor}[2]{\not{#1}#2 \lor #1\not{#2}}
\tikzstyle{every picture}+=[remember picture]

\begin{enumerate}[label={[OH\arabic*]},start=3]
    \item Die Funktion $f(x,y,z) = \left(\not{x}\not{y} + \not{x}y + \not{x}\not{y}\not{z},~~xy + \not{x}\not{y}\not{z}\right)$ kann mittels dieses normierten PLA realisiert werden:
        \begin{center}
            \begin{tikzpicture}
                % AND Block
                    % Boxes
                    \node[draw, minimum width=1cm, minimum height=1cm] (0-0) {3};
                    \node[draw, minimum width=1cm, minimum height=1cm] (1-0) [below =of 0-0] {3};
                    \node[draw, minimum width=1cm, minimum height=1cm] (2-0) [below =of 1-0] {0};

                    \node[draw, minimum width=1cm, minimum height=1cm] (0-1) [right =of 0-0] {3};
                    \node[draw, minimum width=1cm, minimum height=1cm] (1-1) [below =of 0-1] {2};
                    \node[draw, minimum width=1cm, minimum height=1cm] (2-1) [below =of 1-1] {0};

                    \node[draw, minimum width=1cm, minimum height=1cm] (0-2) [right =of 0-1] {2};
                    \node[draw, minimum width=1cm, minimum height=1cm] (1-2) [below =of 0-2] {2};
                    \node[draw, minimum width=1cm, minimum height=1cm] (2-2) [below =of 1-2] {0};

                    \node[draw, minimum width=1cm, minimum height=1cm] (0-3) [right =of 0-2] {3};
                    \node[draw, minimum width=1cm, minimum height=1cm] (1-3) [below =of 0-3] {3};
                    \node[draw, minimum width=1cm, minimum height=1cm] (2-3) [below =of 1-3] {3};

                    % Arrows
                    \draw[{Latex}-] (0-0.north) -- ($(0-0)+(0,1)$) node[above] (nestart) {$1$};
                    \draw[{Latex}-] (0-1.north) -- ($(0-1)+(0,1)$) node[above] {$1$};
                    \draw[{Latex}-] (0-2.north) -- ($(0-2)+(0,1)$) node[above] {$1$};
                    \draw[{Latex}-] (0-3.north) -- ($(0-3)+(0,1)$) node[above] (neend) {$1$};

                    \draw[{Latex}-] (0-0.west) -- ($(0-0)+(-2,0)$) node[above] {$x$};
                        \draw[-{Latex}] (0-0.east) -- node[above] {$x$} (0-1.west);
                        \draw[-{Latex}] (0-1.east) -- node[above] {$x$} (0-2.west);
                        \draw[-{Latex}] (0-2.east) -- node[above] {$x$} (0-3.west);
                        \draw[-{Latex}] (0-3.east) -- node[right] {$~~x$} ($(0-3)+(1,0)$);
                    \draw[{Latex}-] (1-0.west) -- ($(1-0)+(-2,0)$) node[above] {$y$};
                        \draw[-{Latex}] (1-0.east) -- node[above] {$y$} (1-1.west);
                        \draw[-{Latex}] (1-1.east) -- node[above] {$y$} (1-2.west);
                        \draw[-{Latex}] (1-2.east) -- node[above] {$y$} (1-3.west);
                        \draw[-{Latex}] (1-3.east) -- node[right] {$~~y$} ($(1-3)+(1,0)$);
                    \draw[{Latex}-] (2-0.west) -- ($(2-0)+(-2,0)$) node[above] {$z$};
                        \draw[-{Latex}] (2-0.east) -- node[above] {$z$} (2-1.west);
                        \draw[-{Latex}] (2-1.east) -- node[above] {$z$} (2-2.west);
                        \draw[-{Latex}] (2-2.east) -- node[above] {$z$} (2-3.west);
                        \draw[-{Latex}] (2-3.east) -- node[right] {$~~z$} ($(2-3)+(1,0)$);

                % OR Block
                    \node[draw, minimum width=1cm, minimum height=1cm] (3-0) [below =of 2-0] {1};
                    \node[draw, minimum width=1cm, minimum height=1cm] (4-0) [below =of 3-0] {0};

                    \node[draw, minimum width=1cm, minimum height=1cm] (3-1) [below =of 2-1] {1};
                    \node[draw, minimum width=1cm, minimum height=1cm] (4-1) [below =of 3-1] {0};

                    \node[draw, minimum width=1cm, minimum height=1cm] (3-2) [below =of 2-2] {0};
                    \node[draw, minimum width=1cm, minimum height=1cm] (4-2) [below =of 3-2] {1};

                    \node[draw, minimum width=1cm, minimum height=1cm] (3-3) [below =of 2-3] {1};
                    \node[draw, minimum width=1cm, minimum height=1cm] (4-3) [below =of 3-3] {1};

                    \draw[{Latex}-] (3-0.west) -- ($(3-0)+(-1,0)$) node[left] (gsestart) {$0$};
                    \draw[{Latex}-] (4-0.west) -- ($(4-0)+(-1,0)$) node[left] (gseend) {$0$};

                % Down Arrows
                    \draw[-{Latex}] (0-0.south) -- node[right] {$\not{x}$} (1-0.north);
                    \draw[-{Latex}] (1-0.south) -- node[right] {$\not{x}\not{y}$} (2-0.north);
                    \draw[-{Latex}] (2-0.south) -- node[right] {$\not{x}\not{y}$} (3-0.north);
                    \draw[-{Latex}] (3-0.south) -- node[right] {$\not{x}\not{y}$} (4-0.north);
                    \draw[-{Latex}] (4-0.south) -- ($(4-0)+(0,-1)$) node[below] {$\not{x}\not{y}$};

                    \draw[-{Latex}] (0-1.south) -- node[right] {$\not{x}$} (1-1.north);
                    \draw[-{Latex}] (1-1.south) -- node[right] {$\not{x}y$} (2-1.north);
                    \draw[-{Latex}] (2-1.south) -- node[right] {$\not{x}y$} (3-1.north);
                    \draw[-{Latex}] (3-1.south) -- node[right] {$\not{x}y$} (4-1.north);
                    \draw[-{Latex}] (4-1.south) -- ($(4-1)+(0,-1)$) node[below] {$\not{x}y$};

                    \draw[-{Latex}] (0-2.south) -- node[right] {$x$} (1-2.north);
                    \draw[-{Latex}] (1-2.south) -- node[right] {$xy$} (2-2.north);
                    \draw[-{Latex}] (2-2.south) -- node[right] {$xy$} (3-2.north);
                    \draw[-{Latex}] (3-2.south) -- node[right] {$xy$} (4-2.north);
                    \draw[-{Latex}] (4-2.south) -- ($(4-2)+(0,-1)$) node[below] {$xy$};

                    \draw[-{Latex}] (0-3.south) -- node[right] {$\not{x}$} (1-3.north);
                    \draw[-{Latex}] (1-3.south) -- node[right] {$\not{x}\not{y}$} (2-3.north);
                    \draw[-{Latex}] (2-3.south) -- node[right] {$\not{x}\not{y}\not{z}$} (3-3.north);
                    \draw[-{Latex}] (3-3.south) -- node[right] {$\not{x}\not{y}\not{z}$} (4-3.north);
                    \draw[-{Latex}] (4-3.south) -- ($(4-3)+(0,-1)$) node[below] {$\not{x}\not{y}\not{z}$};

                % Outputs
                    \draw[-{Latex}] (3-0.east) -- node[above] {\scriptsize $\not{x}\not{y}$} (3-1.west);
                    \draw[-{Latex}] (3-1.east) -- node[above] {\scriptsize $\not{x}\not{y}$} node[below] {\scriptsize $+ \not{x}y$} (3-2.west);
                    \draw[-{Latex}] (3-2.east) -- node[above] {\scriptsize $\not{x}\not{y}$} node[below] {\scriptsize $+ \not{x}y$} (3-3.west);
                    \draw[-{Latex}] (3-3.east) -- ($(3-3)+(2,0)$) node[above] {\hspace{1em}$\not{x}\not{y}+\not{x}y+\not{x}\not{y}\not{z}$};

                    \draw[-{Latex}] (4-0.east) -- node[above] {\scriptsize $0$} (4-1.west);
                    \draw[-{Latex}] (4-1.east) -- node[above] {\scriptsize $0$} (4-2.west);
                    \draw[-{Latex}] (4-2.east) -- node[above] {\scriptsize $xy$}(4-3.west);
                    \draw[-{Latex}] (4-3.east) -- ($(4-3)+(2,0)$) node[above] {$xy + \not{x}\not{y}\not{z}$};

                % Labels
                    % https://tex.stackexchange.com/a/8906 Box
                    % https://tex.stackexchange.com/a/71479 Middle Pos 
                    \draw[draw=red,thick,dotted] ($(nestart.north west)$) rectangle ($(neend.south east)$) node at ($(nestart)!0.5!(neend)+(0,0.7)$) {\textcolor{red}{neutralisiert}}; % midway

                    \draw[draw=blue,thick,dotted] ($(gsestart.north west)$) rectangle ($(gseend.south east)$) node[rotate=90] at ($(gsestart)!0.5!(gseend)+(-0.7,0)$) {\textcolor{blue}{gesperrt}};

                    \draw[draw=OliveGreen,fill=OliveGreen,fill opacity=0.3,thick,dashed] ($(0-0.north west)+(-0.3,0.3)$) rectangle ($(2-3.south east)+(0.3,-0.3)$) ($(0-0)+(-0.8,1)$) node[left,fill opacity=1] {\textcolor{OliveGreen}{Und-Ebene}};

                    \draw[draw=Salmon,fill=Salmon,fill opacity=0.3,thick,dashed] ($(3-0.north west)+(-0.3,0.3)$) rectangle ($(4-3.south east)+(0.3,-0.3)$) ($(4-3)-(-3,1)$) node[left,fill opacity=1] {\textcolor{Salmon}{Oder-Ebene}};
            \end{tikzpicture}
        \end{center}
    \newpage
    \item \blanko
        \begin{center}
            \renewcommand*{\arraystretch}{1.6}
            \begin{tabular}{lp{9cm}}
                \toprule
                Gemeinsamkeit & Sowohl die Milka ChocoWafer als auch der Siliziumwafer haben die Form einer Scheibe. \\
                \midrule
                \makecell[bl]{Unterschiede \\ (als Wafer)} &  Eine Milka ChocoWafer ist von vielen Schichten aus unterschiedlichen Zutaten aufgebaut, aber ein roher Siliziumwafer hat nur eine Siliziumschicht (der Wafer selbst). \\
                & Geometrisch ist eine Milka ChocoWafer im Vergleich zu einem Siliziumwafer proportional viel dicker. \\
                \makecell[bl]{ \\ (als Wafer mit Chips)} & Als fertigen Chips betrachtet kann die unterste Schokoladeschicht einer Milka ChocoWafer mit der Siliziumschicht (der Wafer selbst) eines fertigen Chips verglichen werden. Hier ist die Milka ChocoWafer mit Schokolade eingekapselt, aber bei einem fertigen Chip entspricht das Silizium nur die unterste Schicht.\\
                & Ferner gibt es bei fertigen Chips viele einzelne Verbindungen zwischen der verschiedenen Schichten aus senkrecht laufende Drahte, sodass ein "PLA" realisiert werden kann. In einer Milka ChocoWafer sind die verschiedene Wafer-Schichten mit Schokolade ohne jegliche Struktur einfach getrennt. Die einzelne Wafer-Schichten können also nicht miteinander "kommunizieren". \\
                \bottomrule
            \end{tabular}
        \end{center}
        \vspace{\baselineskip}
\end{enumerate}