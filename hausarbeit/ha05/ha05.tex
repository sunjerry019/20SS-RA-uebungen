\newcommand{\boxy}[2][yellow]{\mathchoice%
  {\pgfsetfillopacity{0.3}\colorbox{#1}{\pgfsetfillopacity{1}$\displaystyle#2$}}%
  {\pgfsetfillopacity{0.3}\colorbox{#1}{\pgfsetfillopacity{1}$\textstyle#2$}}%
  {\pgfsetfillopacity{0.3}\colorbox{#1}{\pgfsetfillopacity{1}$\scriptstyle#2$}}%
  {\pgfsetfillopacity{0.3}\colorbox{#1}{\pgfsetfillopacity{1}$\scriptscriptstyle#2$}}}%

\let\bnot\xoverline
\let\bnor\downarrow
\newcommand{\xorexpanded}[2]{\bnot{#1}#2 \lor #1\bnot{#2}}
\newcommand{\xor}[0]{\nleftrightarrow}
\newcommand{\linenum}[1]{\textcolor{RedViolet}{\texttt{#1}}}
\tikzstyle{every picture}+=[remember picture]

\begin{enumerate}[label={[OH\arabic*]},start=10]
    \item
        \begin{enumerate}
            \makeatletter
                \setlength{\leftmargins}{\@totalleftmargin}
            \makeatother

            \item 
                \begin{enumerate}
                    \item Zeile \linenum{32}: aktuelle Buchstabe in ein Register geladen
                    \item Zeile \linenum{24,77,81}: Text mit Anfangsaddress $a0$ wird auf der Konsole ausgeben
                    \item Zeile \linenum{43}: Springen, sofern ein Wert größer ist, als der Wert der ASCII-Darstellung des Buchstaben "Z"
                    \item Zeile \linenum{31}: Springen, sofern alle zu verschlüsselnden Buchstaben betrachtet wurden
                \end{enumerate}

            % https://quantixed.org/2018/10/23/new-lexicon-how-to-add-a-custom-minted-lexer-in-overleaf/
            % breaklines
            \item \blanko
                { \small \inputminted[linenos,firstnumber=last,autogobble,xleftmargin=-\leftmargins,frame=leftline,framesep=10pt]{mipslexer.py:MIPSLexer -x}{caeser.s} }
        \end{enumerate}
\end{enumerate}