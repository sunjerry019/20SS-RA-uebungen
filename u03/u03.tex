\def\nrub{3}

\RequirePackage{currfile}
\documentclass[11pt]{article}
\usepackage[utf8]{inputenc}
\usepackage[ngerman]{babel}
\usepackage{libertine}
\usepackage[a4paper,top=1.6in]{geometry}
\usepackage{parskip}
\usepackage{amsmath, amsthm, amssymb} 
\usepackage{mathtools}
\usepackage{booktabs}
\usepackage{tabularx}
\usepackage{enumitem}
\usepackage{graphicx}
\usepackage[table,dvipsnames]{xcolor}
\usepackage{float}
\usepackage{wrapfig}
\usepackage[makeroom]{cancel}
\usepackage{multicol}
\usepackage{multirow}
\usepackage{vwcol} 	 	% Provides variable multicol
\usepackage{commath} 	% Provides good differentials
\usepackage{esint} 		% Provides various fancy integral symbols
\usepackage[binary-units=true]{siunitx} 	% Provides good units
\usepackage{nicefrac}
\usepackage{dashrule}
% \usepackage{minted}
\usepackage{bookmark}
\usepackage{icomma}
\usepackage{siunitx}
\sisetup{
    locale = DE,
    per-mode=fraction,
    fraction-function=\tfrac
}

% https://github.com/mitchpaulus/siunitxext/blob/master/siunitxext.sty
\DeclareSIUnit\inch{in}
\DeclareSIUnit\in{in}
\DeclareSIUnit\pixel{pixel}
\DeclareSIUnit\pix{pixel}

\usepackage[version=0.96]{pgf}
\usepackage{tikz}
\usetikzlibrary{arrows,arrows.meta,shapes,snakes,automata,backgrounds,petri,positioning,circuits.logic.US}

\usepackage{csquotes}
\MakeOuterQuote{"}

% https://texfaq.org/FAQ-twooptarg
    % \usepackage{xargs}
    % \usepackage{etoolbox} % https://tex.stackexchange.com/a/53079
    % \usepackage{ifthen}
    % \newboolean{default}
    % \newcommand{\case}{}
    % \newcommand{\default}{}

    % \newenvironment{switch}[1]{%
    %     \setboolean{default}{true}
    %     \renewcommand{\case}[2]{\ifthenelse{\equal{#1}{##1}}{%
    %         \setboolean{default}{false}##2}{}}%
    %     \renewcommand{\default}[1]{\ifthenelse{\boolean{default}}{##1}{}}
    % }{}
% https://tex.stackexchange.com/a/349350

% http://packages.oth-regensburg.de/ctan/macros/latex/contrib/currfile/currfile.pdf
% % https://tex.stackexchange.com/a/54891/116525

\makeatletter
\newcommand{\toleftmargin}[1]{\par\noindent\hspace{-\@totalleftmargin}\parbox[t]{\textwidth}{#1}}
% \newenvironment{javaenv}{\begin{minted}[linenos,firstnumber=last,autogobble,xleftmargin=-\@totalleftmargin]{java}}{}
\newlength{\leftmargins}
\makeatother
% https://tex.stackexchange.com/a/481735


\makeatletter
% the contents of \squarecorner were mostly stolen from pgfmoduleshapes.code.tex
\def\squarecorner#1{
    % Calculate x
    %
    % First, is width < minimum width?
    \pgf@x=\the\wd\pgfnodeparttextbox%
    \pgfmathsetlength\pgf@xc{\pgfkeysvalueof{/pgf/inner xsep}}%
    \advance\pgf@x by 2\pgf@xc%
    \pgfmathsetlength\pgf@xb{\pgfkeysvalueof{/pgf/minimum width}}%
    \ifdim\pgf@x<\pgf@xb%
        % yes, too small. Enlarge...
        \pgf@x=\pgf@xb%
    \fi%
    % Calculate y
    %
    % First, is height+depth < minimum height?
    \pgf@y=\ht\pgfnodeparttextbox%
    \advance\pgf@y by\dp\pgfnodeparttextbox%
    \pgfmathsetlength\pgf@yc{\pgfkeysvalueof{/pgf/inner ysep}}%
    \advance\pgf@y by 2\pgf@yc%
    \pgfmathsetlength\pgf@yb{\pgfkeysvalueof{/pgf/minimum height}}%
    \ifdim\pgf@y<\pgf@yb%
        % yes, too small. Enlarge...
        \pgf@y=\pgf@yb%
    \fi%
    %
    % this \ifdim is the actual part that makes the node dimensions square.
    \ifdim\pgf@x<\pgf@y%
        \pgf@x=\pgf@y%
    \else
        \pgf@y=\pgf@x%
    \fi
    %
    % Now, calculate right border: .5\wd\pgfnodeparttextbox + .5 \pgf@x + #1outer sep
    \pgf@x=#1.5\pgf@x%
    \advance\pgf@x by.5\wd\pgfnodeparttextbox%
    \pgfmathsetlength\pgf@xa{\pgfkeysvalueof{/pgf/outer xsep}}%
    \advance\pgf@x by#1\pgf@xa%
    % Now, calculate upper border: .5\ht-.5\dp + .5 \pgf@y + #1outer sep
    \pgf@y=#1.5\pgf@y%
    \advance\pgf@y by-.5\dp\pgfnodeparttextbox%
    \advance\pgf@y by.5\ht\pgfnodeparttextbox%
    \pgfmathsetlength\pgf@ya{\pgfkeysvalueof{/pgf/outer ysep}}%
    \advance\pgf@y by#1\pgf@ya%
}
\makeatother

\pgfdeclareshape{square}{
    \savedanchor\northeast{\squarecorner{}}
    \savedanchor\southwest{\squarecorner{-}}

    \foreach \x in {east,west} \foreach \y in {north,mid,base,south} {
        \inheritanchor[from=rectangle]{\y\space\x}
    }
    \foreach \x in {east,west,north,mid,base,south,center,text} {
        \inheritanchor[from=rectangle]{\x}
    }
    \inheritanchorborder[from=rectangle]
    \inheritbackgroundpath[from=rectangle]
}
% https://tex.stackexchange.com/a/300130

\newcommand{\blanko}[0]{\textcolor{white}{.}}
\newcommand{\makeset}[1]{\left\{\,#1\,\right\}}
\newcommand*{\sigfig}[1]{\hspace{0.5cm}\text{(#1 sig. Zif.)}}
\newcommand*{\nks}[1]{\hspace{0.5cm}\text{(#1 Nks.)}}
\DeclarePairedDelimiter{\ceil}{\lceil}{\rceil}
\DeclarePairedDelimiter{\floor}{\lfloor}{\rfloor}
% https://tex.stackexchange.com/a/42274/116525

\usepackage{hyperref}
\hypersetup{
	pdftitle={Rechnerarchitektur (SS20) Übungsblatt \nrub ~- 12141043},
	pdfauthor={Yudong Sun},
	bookmarksnumbered=true,
	bookmarksopen=true,
	bookmarksopenlevel=2,
	pdfstartview=Fit,
	pdfpagemode=UseOutlines,
	colorlinks=true,
	linkcolor=black,
	filecolor=magenta,      
	urlcolor=blue
}
\urlstyle{same}

\renewcommand{\ttdefault}{cmtt}

\usepackage{fancyhdr}
 
\pagestyle{fancy}
\fancyhf{}
\fancyhead[RO]{Yudong Sun / \texttt{12141043}}
\fancyhead[LO]{Übungsblatt \nrub}
\fancyhead[LE]{\texttt{12141043} / Yudong Sun}
\fancyhead[RE]{Übungsblatt \nrub}
\cfoot{\thepage}

\title{Rechnerarchitektur (SS20)\\Übungsblatt \nrub}
\author{Yudong Sun\\\texttt{12141043}}
\date{\today}

\begin{document}

\maketitle

% \input{tutorials.tex}
\let\not\overline
\let\nor\downarrow
\newcommand{\xor}[2]{\not{#1}#2 \lor #1\not{#2}}
\begin{enumerate}[label={Aufgabe Ü\arabic*},start=4]
    \item 
    	\begin{enumerate}[label={\alph*.}]
    		\item Gegeben sei $g(a,b,c) = a \lor \bar{b} \lor (a \land c)$.
    			\begin{center}
    				\ttfamily
    				\begin{tabular}{lll | r}
    					\toprule
    					$a$ & $b$ & $c$ & $g$ \\
    					\midrule
    					0 & 0 & 0 & 1\\
    					0 & 0 & 1 & 1\\
    					0 & 1 & 0 & 0\\
    					0 & 1 & 1 & 0\\
    					1 & 0 & 0 & 1\\
    					1 & 0 & 1 & 1\\
    					1 & 1 & 0 & 1\\
    					1 & 1 & 1 & 1\\
    					\bottomrule
    				\end{tabular}
    			\end{center}
    			\vspace{1em}
    		\item \blanko
    			\vspace{-1.8\baselineskip}
    			% Solve with K - Map
    			\begin{flalign}
    				f_1(A,B,C) &= \not{(A \lor B \lor C)} &&\\
    				f_2(A,B,C) &= C\left(\not{\xor{A}{B}}\right) \lor \not{C}\left(\xor{A}{B}\right) && \\
    				f_3(A,B,C) &= \not{B} && \\
    				f_4(A,B,C) &= B\not{C} && \\
    				f_5(A,B,C) &= A \not{A} && \\
    				f_6(A,B,C) &= C 
    			\end{flalign}
    		\item Aus der De-Morgansche Regeln lässt sich das logische \texttt{NOR} als $a \nor b = \not{a \lor b} = \not{a} \land \not{b}$ ausgedrückt werden. Nach Idempotenz von $\lor$ ist $a = a \lor a$ und folglich $\not{a} = \not{a \lor a} = a \nor a$.
    			\begin{align*}
    				h(a,b,c) = (a \land b) \lor c &= \left(\not{\not{a}} \land \not{\not{b}}\right) \lor c = \left(\not{a} \nor \not{b}\right) \lor c \\
    				&= \not{\not{\left(\not{a} \nor \not{b}\right) \lor c}} = \not{\left(\not{a} \nor \not{b}\right) \nor c}\\
    				&= \not{\left(\left[a \nor a\right] \nor \left[b \nor b\right]\right) \nor c} \\
    				&= \left[\left(\left[a \nor a\right] \nor \left[b \nor b\right]\right) \nor c\right] \nor \left[\left(\left[a \nor a\right] \nor \left[b \nor b\right]\right) \nor c\right]
    			\end{align*}
    	\end{enumerate}
    \item \label{auf:mux}
    	Wir erstellen zunächst drei Wahrheitstabelle zu diesem Multiplexer, indem wir die Zeilen so ordnen, dass eine Spalte aus 1 Block von 4 Nullen und 1 Block von 4 Einsen entsteht. $F$ in diesem Fall bedeutet die Ausgabe des Multiplexers.
    	\begin{multicols}{3}
	    	\begin{center}
	    		\ttfamily
	    		\begin{tabular}{ccc|c}
	    			\toprule
	    			$I_0$ & $I_1$ & $S$ & $F$ \\
	    			\midrule
	    			\textcolor{red}{0} & 0 & 0 & 0 \\
					\textcolor{red}{0} & 0 & 1 & 0 \\
					\textcolor{red}{0} & 1 & 0 & 0 \\
					\textcolor{red}{0} & 1 & 1 & 1 \\
					\rowcolor{SpringGreen} \textcolor{red}{1} & 0 & 0 & 1 \\
					\rowcolor{SpringGreen} \textcolor{red}{1} & 0 & 1 & 0 \\
					\textcolor{red}{1} & 1 & 0 & 1 \\
					\textcolor{red}{1} & 1 & 1 & 1 \\
	    			\bottomrule
	    		\end{tabular}
			\end{center}
			\begin{center}
	    		\ttfamily
	    		\begin{tabular}{ccc|c}
	    			\toprule
	    			$I_0$ & $I_1$ & $S$ & $F$ \\
	    			\midrule
	    			0 & \textcolor{red}{0} & 0 & 0 \\
					0 & \textcolor{red}{0} & 1 & 0 \\
					1 & \textcolor{red}{0} & 0 & 1 \\
					1 & \textcolor{red}{0} & 1 & 0 \\
					\rowcolor{yellow} 0 & \textcolor{red}{1} & 0 & 0 \\
					\rowcolor{yellow} 0 & \textcolor{red}{1} & 1 & 1 \\
					\rowcolor{yellow} 1 & \textcolor{red}{1} & 0 & 1 \\
					\rowcolor{yellow} 1 & \textcolor{red}{1} & 1 & 1 \\
	    			\bottomrule
	    		\end{tabular}
			\end{center}
			\begin{center}
	    		\ttfamily
	    		\begin{tabular}{ccc|c}
	    			\toprule
	    			$I_0$ & $I_1$ & $S$ & $F$ \\
	    			\midrule
	    			0 & 0 & \textcolor{red}{0} & 0 \\
					1 & 0 & \textcolor{red}{0} & 1 \\
					0 & 1 & \textcolor{red}{0} & 0 \\
					1 & 1 & \textcolor{red}{0} & 1 \\
					0 & 0 & \textcolor{red}{1} & 0 \\
					0 & 1 & \textcolor{red}{1} & 1 \\
					1 & 0 & \textcolor{red}{1} & 0 \\
					1 & 1 & \textcolor{red}{1} & 1 \\
	    			\bottomrule
	    		\end{tabular}
			\end{center}
	    \end{multicols}
	    Die gelbe Zeile entsprechen die gewünschte Wahrheitstabelle zur \ref{auf:mux}\ref{auf:aub}
	    \begin{multicols}{2}
	    	\noindent
	    	\begin{enumerate}[label={\alph*.}]
	    		\item \label{auf:aub} \blanko
		    		\begin{center}
		    			% https://tex.stackexchange.com/a/428172
		    			\begin{tikzpicture}
		    				\node[draw, minimum width=1cm, minimum height=2cm] (mux) {};
		    				% Input
		    				\draw[{Latex}-] ($(mux.west)+(0,0.5)$) -- ($(mux.west)+(-1.5,0.5)$) node[above right] {A};
		    				\draw[{Latex}-] ($(mux.west)+(0,-0.5)$) -- ($(mux.west)+(-1.5,-0.5)$) node[above right] {1};

		    				% Control
		    				\draw[{Latex}-] (mux.south) -- ($(mux)+(0,-2)$) node[above right] {B};

		    				% Output
		    				\draw[-{Latex}] (mux.east) --  ($(mux)+(1.5,0)$) node[above] {A $\lor$ B};
		    			\end{tikzpicture}
		    		\end{center}
		    	\item \blanko
		    		\label{auf:nota}
		    		\begin{center}
		    			% https://tex.stackexchange.com/a/428172
		    			\begin{tikzpicture}
		    				\node[draw, minimum width=1cm, minimum height=2cm] (mux) {};
		    				% Input
		    				\draw[{Latex}-] ($(mux.west)+(0,0.5)$) -- ($(mux.west)+(-1.5,0.5)$) node[above right] {1};
		    				\draw[{Latex}-] ($(mux.west)+(0,-0.5)$) -- ($(mux.west)+(-1.5,-0.5)$) node[above right] {0};

		    				% Control
		    				\draw[{Latex}-] (mux.south) -- ($(mux)+(0,-2)$) node[above right] {A};

		    				% Output
		    				\draw[-{Latex}] (mux.east) --  ($(mux)+(1.5,0)$) node[above] {$\not{\text{A}}$};
		    			\end{tikzpicture}
		    		\end{center}
	    	\end{enumerate}
	    \end{multicols}
    \item Sehen Sie bitte \texttt{u02-ue06.txt}
\end{enumerate}

\end{document}
