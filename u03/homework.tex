\let\not\overline
\let\nor\downarrow
\newcommand{\xor}[2]{\not{#1}#2 \lor #1\not{#2}}
\begin{enumerate}[label={Aufgabe Ü\arabic*},start=4]
    \item 
    	\begin{enumerate}[label={\alph*.}]
    		\item Gegeben sei $g(a,b,c) = a \lor \bar{b} \lor (a \land c)$.
    			\begin{center}
    				\ttfamily
    				\begin{tabular}{lll | r}
    					\toprule
    					$a$ & $b$ & $c$ & $g$ \\
    					\midrule
    					0 & 0 & 0 & 1\\
    					0 & 0 & 1 & 1\\
    					0 & 1 & 0 & 0\\
    					0 & 1 & 1 & 0\\
    					1 & 0 & 0 & 1\\
    					1 & 0 & 1 & 1\\
    					1 & 1 & 0 & 1\\
    					1 & 1 & 1 & 1\\
    					\bottomrule
    				\end{tabular}
    			\end{center}
    			\vspace{1em}
    		\item \blanko
    			\vspace{-1.8\baselineskip}
    			% Solve with K - Map
    			\begin{flalign}
    				f_1(A,B,C) &= \not{(A \lor B \lor C)} &&\\
    				f_2(A,B,C) &= C\left(\not{\xor{A}{B}}\right) \lor \not{C}\left(\xor{A}{B}\right) && \\
    				f_3(A,B,C) &= \not{B} && \\
    				f_4(A,B,C) &= B\not{C} && \\
    				f_5(A,B,C) &= A \not{A} && \\
    				f_6(A,B,C) &= C 
    			\end{flalign}
    		\item Aus der De-Morgansche Regeln lässt sich das logische \texttt{NOR} als $a \nor b = \not{a \lor b} = \not{a} \land \not{b}$ ausgedrückt werden. Nach Idempotenz von $\lor$ ist $a = a \lor a$ und folglich $\not{a} = \not{a \lor a} = a \nor a$.
    			\begin{align*}
    				h(a,b,c) = (a \land b) \lor c &= \left(\not{\not{a}} \land \not{\not{b}}\right) \lor c = \left(\not{a} \nor \not{b}\right) \lor c \\
    				&= \not{\not{\left(\not{a} \nor \not{b}\right) \lor c}} = \not{\left(\not{a} \nor \not{b}\right) \nor c}\\
    				&= \not{\left(\left[a \nor a\right] \nor \left[b \nor b\right]\right) \nor c} \\
    				&= \left[\left(\left[a \nor a\right] \nor \left[b \nor b\right]\right) \nor c\right] \nor \left[\left(\left[a \nor a\right] \nor \left[b \nor b\right]\right) \nor c\right]
    			\end{align*}
    	\end{enumerate}
    \item \label{auf:mux}
    	Wir erstellen zunächst drei Wahrheitstabelle zu diesem Multiplexer, indem wir die Zeilen so ordnen, dass eine Spalte aus 1 Block von 4 Nullen und 1 Block von 4 Einsen entsteht. $F$ in diesem Fall bedeutet die Ausgabe des Multiplexers.
    	\begin{multicols}{3}
	    	\begin{center}
	    		\ttfamily
	    		\begin{tabular}{ccc|c}
	    			\toprule
	    			$I_0$ & $I_1$ & $S$ & $F$ \\
	    			\midrule
	    			\textcolor{red}{0} & 0 & 0 & 0 \\
					\textcolor{red}{0} & 0 & 1 & 0 \\
					\textcolor{red}{0} & 1 & 0 & 0 \\
					\textcolor{red}{0} & 1 & 1 & 1 \\
					\rowcolor{SpringGreen} \textcolor{red}{1} & 0 & 0 & 1 \\
					\rowcolor{SpringGreen} \textcolor{red}{1} & 0 & 1 & 0 \\
					\textcolor{red}{1} & 1 & 0 & 1 \\
					\textcolor{red}{1} & 1 & 1 & 1 \\
	    			\bottomrule
	    		\end{tabular}
			\end{center}
			\begin{center}
	    		\ttfamily
	    		\begin{tabular}{ccc|c}
	    			\toprule
	    			$I_0$ & $I_1$ & $S$ & $F$ \\
	    			\midrule
	    			0 & \textcolor{red}{0} & 0 & 0 \\
					0 & \textcolor{red}{0} & 1 & 0 \\
					1 & \textcolor{red}{0} & 0 & 1 \\
					1 & \textcolor{red}{0} & 1 & 0 \\
					\rowcolor{yellow} 0 & \textcolor{red}{1} & 0 & 0 \\
					\rowcolor{yellow} 0 & \textcolor{red}{1} & 1 & 1 \\
					\rowcolor{yellow} 1 & \textcolor{red}{1} & 0 & 1 \\
					\rowcolor{yellow} 1 & \textcolor{red}{1} & 1 & 1 \\
	    			\bottomrule
	    		\end{tabular}
			\end{center}
			\begin{center}
	    		\ttfamily
	    		\begin{tabular}{ccc|c}
	    			\toprule
	    			$I_0$ & $I_1$ & $S$ & $F$ \\
	    			\midrule
	    			0 & 0 & \textcolor{red}{0} & 0 \\
					1 & 0 & \textcolor{red}{0} & 1 \\
					0 & 1 & \textcolor{red}{0} & 0 \\
					1 & 1 & \textcolor{red}{0} & 1 \\
					0 & 0 & \textcolor{red}{1} & 0 \\
					0 & 1 & \textcolor{red}{1} & 1 \\
					1 & 0 & \textcolor{red}{1} & 0 \\
					1 & 1 & \textcolor{red}{1} & 1 \\
	    			\bottomrule
	    		\end{tabular}
			\end{center}
	    \end{multicols}
	    Die gelbe Zeile entsprechen die gewünschte Wahrheitstabelle zur \ref{auf:mux}\ref{auf:aub}
	    \begin{multicols}{2}
	    	\noindent
	    	\begin{enumerate}[label={\alph*.}]
	    		\item \label{auf:aub} \blanko
		    		\begin{center}
		    			% https://tex.stackexchange.com/a/428172
		    			\begin{tikzpicture}
		    				\node[draw, minimum width=1cm, minimum height=2cm] (mux) {};
		    				% Input
		    				\draw[{Latex}-] ($(mux.west)+(0,0.5)$) -- ($(mux.west)+(-1.5,0.5)$) node[above right] {A};
		    				\draw[{Latex}-] ($(mux.west)+(0,-0.5)$) -- ($(mux.west)+(-1.5,-0.5)$) node[above right] {1};

		    				% Control
		    				\draw[{Latex}-] (mux.south) -- ($(mux)+(0,-2)$) node[above right] {B};

		    				% Output
		    				\draw[-{Latex}] (mux.east) --  ($(mux)+(1.5,0)$) node[above] {A $\lor$ B};
		    			\end{tikzpicture}
		    		\end{center}
		    	\item \blanko
		    		\label{auf:nota}
		    		\begin{center}
		    			% https://tex.stackexchange.com/a/428172
		    			\begin{tikzpicture}
		    				\node[draw, minimum width=1cm, minimum height=2cm] (mux) {};
		    				% Input
		    				\draw[{Latex}-] ($(mux.west)+(0,0.5)$) -- ($(mux.west)+(-1.5,0.5)$) node[above right] {1};
		    				\draw[{Latex}-] ($(mux.west)+(0,-0.5)$) -- ($(mux.west)+(-1.5,-0.5)$) node[above right] {0};

		    				% Control
		    				\draw[{Latex}-] (mux.south) -- ($(mux)+(0,-2)$) node[above right] {A};

		    				% Output
		    				\draw[-{Latex}] (mux.east) --  ($(mux)+(1.5,0)$) node[above] {$\not{\text{A}}$};
		    			\end{tikzpicture}
		    		\end{center}
	    	\end{enumerate}
	    \end{multicols}
    \item Sehen Sie bitte \texttt{u02-ue06.txt}
\end{enumerate}